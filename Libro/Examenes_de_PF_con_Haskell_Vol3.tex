% Examenes_de_PF_con_Haskell_Vol3.tex
% Exámenes de programación funcional con Haskell.
% José A. Alonso Jiménez <jalonso@us.es>
% Sevilla, 10 de diciembre de 2012 (3 de febrero de 2019)
% =============================================================================

\documentclass[a4paper,12pt,twoside]{book}

%%%%%%%%%%%%%%%%%%%%%%%%%%%%%%%%%%%%%%%%%%%%%%%%%%%%%%%%%%%%%%%%%%%%%%%%%%%%%%
%% § Paquetes adicionales                                                   %%
%%%%%%%%%%%%%%%%%%%%%%%%%%%%%%%%%%%%%%%%%%%%%%%%%%%%%%%%%%%%%%%%%%%%%%%%%%%%%%

% Configuración para XeLaTeX
\usepackage{fontspec}
\usepackage{xltxtra}
\defaultfontfeatures{Ligatures=TeX,Numbers=OldStyle}
\setromanfont{DejaVu Sans}
% \setsansfont{Arial}
\setmonofont{DejaVu Sans Mono}[Scale={0.90}]

% Notas: La lista de fuentes disponibles se obtiene con fc-list

% \usepackage{ucs}
% \usepackage[utf8]{inputenc}        % Acentos de UTF8
\usepackage[spanish]{babel}        % Castellanización.
% \usepackage[T1]{fontenc}           % Codificación T1 con European Computer
%                                    % Modern.  
% \usepackage{graphicx}
\usepackage{fancyvrb}              % Verbatim extendido
% \usepackage{mathpazo}              % Fuentes semejante a palatino
% \usepackage[scaled=.90]{helvet}
% \usepackage{cmtt}
% \renewcommand{\ttdefault}{cmtt}
\usepackage{a4wide}
\usepackage{minted}

\linespread{1.05}                  % Distancia entre líneas
\setlength{\parindent}{2em}        % Indentación de comienzo de párrafo
% \deactivatetilden                  % Elima uso de ~ para la eñe
\raggedbottom                      % No ajusta los espacios verticales.

\usepackage[%
  colorlinks=true,
  urlcolor=blue,
  % pdftex,
  pdfauthor={José A. Alonso <jalonso@us.es>},%
  pdftitle={Examenes de Informatica de 1 de Matematicas},%
  pdfstartview=FitH,%
  bookmarks=false]{hyperref}      

\setcounter{tocdepth}{1}
\setcounter{secnumdepth}{4}

\usepackage{tocstyle}
\usetocstyle{KOMAlike}

\newenvironment{enumerate*}%
  {\vspace*{-0mm}
   \begin{enumerate}%
    \setlength{\itemsep}{0pt}%
    \setlength{\parskip}{0pt}}%
  {\vspace*{-0mm}
   \end{enumerate}}

\newenvironment{itemize*}%
  {\vspace*{-0mm}
   \begin{itemize}%
    \setlength{\itemsep}{0pt}%
    \setlength{\parskip}{0pt}}%
  {\vspace*{-0mm}
   \end{itemize}}

%%%%%%%%%%%%%%%%%%%%%%%%%%%%%%%%%%%%%%%%%%%%%%%%%%%%%%%%%%%%%%%%%%%%%%%%%%%%%%
%% § Cabeceras                                                              %%
%%%%%%%%%%%%%%%%%%%%%%%%%%%%%%%%%%%%%%%%%%%%%%%%%%%%%%%%%%%%%%%%%%%%%%%%%%%%%%

\usepackage{fancyhdr}

\addtolength{\headheight}{\baselineskip}

\pagestyle{fancy}

\cfoot{}

\fancyhead{}
\fancyhead[RE]{\mdseries\sffamily \nouppercase{\leftmark}}
\fancyhead[LO]{\mdseries\sffamily \nouppercase{\rightmark}}
\fancyhead[LE,RO]{\mdseries\sffamily \thepage}

%%%%%%%%%%%%%%%%%%%%%%%%%%%%%%%%%%%%%%%%%%%%%%%%%%%%%%%%%%%%%%%%%%%%%%%%%%%%%%
%% § Definiciones                                                           %%
%%%%%%%%%%%%%%%%%%%%%%%%%%%%%%%%%%%%%%%%%%%%%%%%%%%%%%%%%%%%%%%%%%%%%%%%%%%%%%

\input definiciones

%%%%%%%%%%%%%%%%%%%%%%%%%%%%%%%%%%%%%%%%%%%%%%%%%%%%%%%%%%%%%%%%%%%%%%%%%%%%%%
%% § Título                                                                 %%
%%%%%%%%%%%%%%%%%%%%%%%%%%%%%%%%%%%%%%%%%%%%%%%%%%%%%%%%%%%%%%%%%%%%%%%%%%%%%%

\title{
  {\LARGE Exámenes de \\ ``Programaci\'on funcional con Haskell''} \\ 
  {\large Vol. 3 (Curso 2011--12)}} 
\author{
  \href{http://www.cs.us.es/~jalonso}{José A. Alonso Jiménez}}
\date{\vfill \hrule \vspace*{2mm}
  \begin{tabular}{l}
      \href{http://www.cs.us.es/glc}
           {Grupo de Lógica Computacional} \\
      \href{http://www.cs.us.es}
           {Dpto. de Ciencias de la Computación e Inteligencia Artificial} \\
      \href{http://www.us.es}
           {Universidad de Sevilla}  \\
      Sevilla, 10 de diciembre de 2012
  \end{tabular}\hfill\mbox{}}

%%%%%%%%%%%%%%%%%%%%%%%%%%%%%%%%%%%%%%%%%%%%%%%%%%%%%%%%%%%%%%%%%%%%%%%%%%%%%%
%% § Documento                                                              %%
%%%%%%%%%%%%%%%%%%%%%%%%%%%%%%%%%%%%%%%%%%%%%%%%%%%%%%%%%%%%%%%%%%%%%%%%%%%%%%

% \includeonly{}

% \includexmp{licencia}

\begin{document}

\maketitle
\newpage

\input{Licencia/licenciaCC}
\newpage

\tableofcontents
\clearpage

\renewcommand{\chaptername}{}

\chapter*{Introducción}
\addcontentsline{toc}{chapter}{Introducción}

Este libro es una recopilación de las soluciones de
ejercicios de los exámenes de programación funcional con Haskell de la
\href{http://www.cs.us.es/~jalonso/cursos/i1m-11}
     {asignatura de Informática (curso 2011--12)}
del
\href{http://www.matematicas.us.es/estudios/grado-en-matematicas}
     {Grado en Matemática} 
de la 
\href{http://www.us.es/}
     {Universidad de Sevilla}.

Los exámenes se realizaron en el aula de informática y su duración
fue de 2 horas. La materia de cada examen es la impartida desde el
comienzo del curso (generalmente, el 1 de octubre) hasta la fecha
del examen. Dicha materia se encuentra en los libros de temas y
ejercicios del curso:
\begin{itemize}
\item
  \href{https://www.cs.us.es/~jalonso/cursos/i1m-11/temas/2011-12-IM-temas-PF.pdf}
  {Temas de programación funcional (curso 2011--12)}\
  \footnote{\url{https://www.cs.us.es/~jalonso/cursos/i1m-11/temas/2011-12-IM-temas-PF.pdf}} 
\item
  \href{https://www.cs.us.es/~jalonso/cursos/i1m-11/ejercicios/ejercicios-I1M-2010.pdf}
  {Ejercicios de ``Informática de 1º de Matemáticas'' (2011--12)}\
  \footnote{\url{https://www.cs.us.es/~jalonso/cursos/i1m-11/ejercicios/ejercicios-I1M-2011.pdf}}
\item
  \href{http://www.cs.us.es/~jalonso/publicaciones/Piensa_en_Haskell.pdf}
  {Piensa en Haskell (Ejercicios de programación funcional con Haskell)}\
  \footnote{\url{http://www.cs.us.es/~jalonso/publicaciones/Piensa_en_Haskell.pdf}}
\end{itemize}

El libro consta de 2 capítulos correspondientes a 2 grupos de la
asignatura. En cada capítulo hay una sección por cada uno de los
exámenes del grupo. Los ejercicios de cada examen han sido propuestos
por los profesores de su grupo (cuyos nombres aparecen en el título del
capítulo). Sin embargo, los he modificado para unificar el estilo de su
presentación.

Finalmente, el libro contiene dos apéndices. Uno con el método de Polya
de resolución de problemas (sobre el que se hace énfasis durante todo el
curso) y el otro con un resumen de las funciones de Haskell de uso más
frecuente.

Los códigos del libro están disponibles en
\href{https://github.com/jaalonso/Examenes_de_PF_con_Haskell_Vol3}
     {GitHub}
     \footnote{{\url{https://github.com/jaalonso/Examenes_de_PF_con_Haskell_Vol3}}}

Este libro es el tercer volumen de la serie de recopilaciones de
exámenes de programación funcional con Haskell. Los volúmenes anteriores son
\begin{itemize}
\item
  \href{https://github.com/jaalonso/Examenes_de_PF_con_Haskell_Vol1}
  {Exámenes de ``Programaci\'on funcional con Haskell''.
    Vol. 1 (Curso 2009--10)}\
    \footnote{\url{{https://github.com/jaalonso/Examenes_de_PF_con_Haskell_Vol1}}}

\item
  \href{https://github.com/jaalonso/Examenes_de_PF_con_Haskell_Vol2}
  {Exámenes de ``Programaci\'on funcional con Haskell''.
    Vol. 1 (Curso 2010--11)}\
    \footnote{\url{{https://github.com/jaalonso/Examenes_de_PF_con_Haskell_Vol2}}}

\end{itemize}
     
\begin{flushright}
  José A. Alonso \\
  Sevilla, 10 de diciembre de 2012
\end{flushright}

\chapter{Exámenes del grupo 1}
\chapterauthor{José A. Alonso y Agustín Riscos}
\section{Examen 1 (26 de Octubre de 2011)}
\examen{Grupo_1/examen_1_26_Oct.hs}
\section{Examen 2 (30 de Noviembre de 2011)}
\examen{Grupo_1/examen_2_30_Nov.hs}
\section{Examen 3 (25 de Enero de 2012)}
\examen{Grupo_1/examen_3_25_Ene.hs}
\section{Examen 4 (29 de Febrero de 2012)}
\examen{Grupo_1/examen_4_29_Feb.hs}
\section{Examen 5 (21 de Marzo de 2012)}
\examen{Grupo_1/examen_5_21_Mar.hs}
\section{Examen 6 ( 2 de Mayo de 2012)}
\examen{Grupo_1/examen_6_02_May.hs}
\section{Examen 7 (25 de Junio de 2012)} 
\examen{Grupo_1/examen_7_25_Jun.hs}
\section{Examen 8 (29 de Junio de 2012)} 
\label{examen_11_12_1_8}
\examen{Grupo_1/examen_8_29_Jun.hs}
\section{Examen 9 ( 9 de Septiembre de 2012)} 
\label{examen_11_12_1_9}
\examen{Grupo_1/examen_9_10_Sep.hs}
\section{Examen 10 (10 de Diciembre de 2012)} 
\label{examen_11_12_1_10}
\examen{Grupo_1/examen_10_10_Dic.hs}

\chapter{Exámenes del grupo 2}
\chapterauthor{María J. Hidalgo}
\section{Examen 1 (27 de Octubre de 2011)}
\examen{Grupo_2/examen_1_27_oct.hs}
\section{Examen 2 ( 1 de Diciembre de 2011)}
\examen{Grupo_2/examen_2_01_dic.hs}
\section{Examen 3 (26 de Enero de 2012)}
\examen{Grupo_2/examen_3_26_ene.hs}
\section{Examen 4 ( 1 de Marzo de 2012)}
\examen{Grupo_2/examen_4_01_mar.hs}
\section{Examen 5 (22 de Marzo de 2012)}
\examen{Grupo_2/examen_5_22_mar.hs}
\section{Examen 6 ( 3 de Mayo de 2012)}
\examen{Grupo_2/examen_6_03_may.hs}
\section{Examen 7 (24 de Junio de 2012)}
\examen{Grupo_2/examen_7_24_jun.hs}
\section{Examen 8 (29 de Junio de 2012)} 
El examen es común con el del grupo 1 (ver página \pageref{examen_11_12_1_8}).
\section{Examen 9 ( 9 de Septiembre de 2012)} 
El examen es común con el del grupo 1 (ver página \pageref{examen_11_12_1_9}).
\section{Examen 10 (10 de Diciembre de 2012)} 
El examen es común con el del grupo 1 (ver página \pageref{examen_11_12_1_10}).

\chapter{Exámenes del grupo 3}
\chapterauthor{Antonia M. Chávez}
\section{Examen 1 (14 de Noviembre de 2011)}
\examen{Grupo_3/examen_1_14_nov.hs}
\section{Examen 2 (12 de Diciembre de 2011)}
\examen{Grupo_3/examen_2_12_dic.hs}
\section{Examen 7 (29 de Junio de 2012)} 
El examen es común con el del grupo 1 (ver página \pageref{examen_11_12_1_8}).
\section{Examen 8 (9 de Septiembre de 2012)} 
El examen es común con el del grupo 1 (ver página \pageref{examen_11_12_1_9}).
\section{Examen 9 (10 de Diciembre de 2012)} 
El examen es común con el del grupo 1 (ver página \pageref{examen_11_12_1_10}).

\chapter{Exámenes del grupo 4}
\chapterauthor{José F. Quesada}
\section{Examen 1 ( 7 de Noviembre de 2011)}
\examen{Grupo_4/examen_1_07_nov.hs}
\section{Examen 2 (30 de Noviembre de 2011)}
\examen{Grupo_4/examen_2_30_nov.hs}
\section{Examen 3 (16 de Enero de 2012)}
\examen{Grupo_4/examen_3_16_ene.hs}
\section{Examen 4 ( 7 de Marzo de 2012)}
\examen{Grupo_4/examen_4_07_mar.hs}
\section{Examen 5 (28 de Marzo de 2012)}
\examen{Grupo_4/examen_5_28_mar.hs}
\section{Examen 6 ( 9 de Mayo de 2012)}
\examen{Grupo_4/examen_6_09_may.hs}
\section{Examen 7 (11 de Junio de 2012)}
\examen{Grupo_4/examen_7_11_jun.hs}
\section{Examen 8 (29 de Junio de 2012)} 
El examen es común con el del grupo 1 (ver página \pageref{examen_11_12_1_8}).
\section{Examen 9 ( 9 de Septiembre de 2012)} 
El examen es común con el del grupo 1 (ver página \pageref{examen_11_12_1_9}).
\section{Examen 10 (10 de Diciembre de 2012)} 
El examen es común con el del grupo 1 (ver página \pageref{examen_11_12_1_10}).

\appendix % Apéndices

% A. Resumen de funciones Haskell
\include{Apendice/resumen_Haskell}

% B. Método de Pólya para la resolución de problemas
\include{Apendice/metodo_de_Polya}

%%%%%%%%%%%%%%%%%%%%%%%%%%%%%%%%%%%%%%%%%%%%%%%%%%%%%%%%%%%%%%%%%%%%%%%%%%%%%%%
%%  Bibliografía                                                            %%
%%%%%%%%%%%%%%%%%%%%%%%%%%%%%%%%%%%%%%%%%%%%%%%%%%%%%%%%%%%%%%%%%%%%%%%%%%%%%%%

% \nocite{Alonso-12a}
\nocite{Alonso-12b}
\nocite{Bird-99a}
\nocite{Cunningham-10a}
\nocite{Daume-06}
\nocite{Davie-92a}
\nocite{Doets-04a}
\nocite{Fokker-96}
\nocite{Hudak-00a}
\nocite{Hudak-12a}
\nocite{Hutton-07a}
\nocite{OSullivan-08a}
\nocite{Rabhi-99a}
\nocite{Polya-65a}
\nocite{Ruiz-04}
\nocite{Thompson-11a}

\addcontentsline{toc}{chapter}{Bibliografía}
\bibliographystyle{abbrv}
\bibliography{Examenes_de_PF_con_Haskell_Vol3}

\end{document}

%%% Local Variables: 
%%% mode: latex
%%% TeX-master: t
%%% End: 

